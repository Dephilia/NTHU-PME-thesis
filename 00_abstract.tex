\begin{abstractzh}
\setcounter{page}{1}

在現代農業中,仰賴人工澆水會增加生產成本且耗費人力,使用溫室噴霧系統可以解決前述問題,卻會造成葉面浸潤,進而增加植物的疾病發生率。並且兩種方法均無法針對個別植株,控制其所需的澆水量以及澆水方向。

本研究研發一新型多致動龍門型澆水機器人,可應用於蝴蝶蘭植株栽培溫室中。首先藉由應用YOLOv4-tiny物件辨識技術,偵測植株位置,再經由計算圖像回歸直線以估測植株姿態。並且利用變形路徑演算法進行路徑規劃,以降低姿態估測的不確定性。最後透過控制龍門上的線性致動器,使致動器沿著規劃路徑移動,藉以避開葉面區域。若實際在致動器上安裝澆水器,而非實驗時使用雷射作為驗證,可望減少澆水時葉面上積水,降低植株病害的可能性。


本研究建構於機器人系統(ROS)上,並且透過Nvidia Xavier實現演算法。最終,與過往的研究比較,本研究利用雷射模擬澆水路徑,實現了82\%的澆水成功率,根據使用者設定速度,可以達到每秒3個植株以上的速度進行精準澆水。


\Keywordszh 龍門機器人、灌溉系統、智慧農業、姿態估測、路徑規劃、機器人作業系統

\end{abstractzh}


\begin{abstracten}

For watering plants, manual watering requires human resources and thus increase cost, while the overhead irrigation system leads to infiltration on the leaf surface, resulting in more unnecessary incidence rate of plants that may cause diseases on the plant. Both methods above cannot be used to precisely control the watering amount and watering direction on each plant.

This research develops a new type of gantry robot to water Phalaenopsis orchid seedlings in the greenhouse. The neural network model YOLOv4-tiny is employed to locate the orchid seedlings and to regress the cropped image to estimate the poses of the seedlings. Furthermore, the reactive deformation trajectory planning algorithm is utilized to generate the sprayer trajectory from the estimated poses of seedlings by YOLOv4-tiny and to reduce the uncertainty of pose estimation. Finally, multiple linear actuators are guided along the planned paths that is supposed to water the soil portion, rather than the leaf area of plants, if the water sprayer is built, instead of using laser beam in this work, in the future. Due to less watering in the leave, the diseases of seedlings can be reduced.

The gantry robot is implemented on the robot operating system (ROS) that is run on an Nvidia Xavier computer. Experiments are conducted and the results are compared to the earlier works, using laser beam to substitute the water sprayer, the proposed system shows that the defined watering success rate can reach 82\% and the defined watering speed is at least three seedlings per second. 

\Keywordsen Gantry robot, Irrigation system, Smart farming, Pose estimation, Path planning, Robot operating system

\end{abstracten}

% \begin{comment}

% \begin{keyword} Object detection, Path planning, Gantry robot, Computer vision, Orchids, Greenhouse, Smart farming, robot operating system
% \end{keyword}
% \end{comment}

%\begin{comment}
%\category{I2.10}{Computing Methodologies}{Artificial Intelligence --
%Vision and Scene Understanding} \category{H5.3}{Information
%Systems}{Information Interfaces and Presentation (HCI) -- Web-based
%Interaction.}
%
%\terms{Design, Human factors, Performance.}
%
%\keywords{Motor control, Path planning, Smart agriculture}
%\end{comment}